
\subsection{Alternative Specifications}
Our analysis may be sensitive to modeling choices such as how to measure democracy and weight countries. Extreme nations like the US and China may also be driving our results. Below we check whether these concerns threaten our findings. 

\textbf{Alternative democracy indices.} 
We adopt alternative democracy indices by the  \citet{centerforsystemicpeacePolity5AnnualTime2018} and the \citet{DemocracyIndex2020}. We present the resulting 2SLS estimates in Appendix Table \ref{tab:2sls-compare-indices}. Using alternative indices does not change our baseline results.

\textbf{Alternative weighting.} Our 2SLS results so far weight countries by GDP. We believe that GDP weighting is reasonable especially when the outcome is the GDP growth rate. 
A possible alternative is to weight countries by population or to not weight countries. We compare our 2SLS results with these different weighting methods in Appendix Table \ref{tab:2sls-compare-weighting}. The qualitative pattern is the same among the three ways to weight countries. 

\textbf{Exclusion of the US and China from the sample.} To check if the US and China drive our results, we show our results without the two countries in Appendix Table \ref{tab:2sls-compare-samples}. We continue to find that democracy significantly causes worse shocks to GDP and higher Covid-19 mortality. % In Panel A, we estimate the effect of democracy on GDP growth. Robust estimates without the US and China range from -5.1 to -7.8 percentage points, while those with the US and China range from -2.8 to -5.5 percentage points. That the coefficients' magnitudes become larger is reasonable since the US GDP growth rate is better than expected. For the estimates for Covid-19-related deaths per million in Panel B, we find that excluding the US and China increases the standard errors, but we continue to find that democracy significantly causes higher mortality. 

Overall, these additional results support the view that democracy plays an important, negative role in economic growth and public health during 2020. 

\subsection{Mechanisms Behind Democracy's Effect}

Does having a stronger democracy cause worse economic and public health outcomes during the Covid pandemic? Media and policy discussions point to the speed, coverage, and severity of containment policies as potential proximate mechanisms. Indeed, Paul Krugman blames ``\textit{catastrophically slow and inadequate}" responses by the US government for its failure.\footnote{Krugman, Paul. 2020. ``3 Rules for the Trump Pandemic." \emph{New York Times.} March 19. \url{https://www.nytimes.com/2020/03/19/opinion/trump-coronavirus.html}} We explore whether this differential in policy responses explain democracy's negative effect we find. Our findings suggest that a key channel for the negative impact of democracy is weaker and narrower containment policies at the beginning of the outbreak. In contrast, the speed of containment policies appears to be less important. 

To measure the severity of policy responses, we use the Containment Health Index at the 10th confirmed case of Covid-19.\footnote{We get similar results when we use the Containment Health Index at the 100th confirmed case or the mean Containment Health Index during 2020. Results are available upon request.} To quantify how widely initial responses cover aspects of civilian life, we look at the percentage of 13 domains in which the government introduced significant containment measures at the 10th confirmed case in the pandemic. The domains are schools, workplaces, public events, gatherings, public transport, stay-at-home requirements, domestic travel, international travel, public information campaigns, testing, contact tracing, facial coverings, and vaccinations. To assess the speed of policy responses, we consider the number of days between the 10th confirmed case and the introduction of any containment policy.\footnote{We get similar results using the number of days between the 10th confirmed case and the introduction of containment policies in each of the 13 domains. We also look at the period between the 100th confirmed case and policy introduction in the mentioned domains, as well as the period between January 1st, 2020 and policy introduction. We continue to get similar results. The results are available upon request.} We determine the introduction date of any containment policy by looking at the date when the Containment Health Index becomes positive.\footnote{All countries begin at zero. The index becomes positive when countries introduce any significant policy to contain Covid-19.}

For each policy response mechanism $M$ (severity, coverage, or speed of containment response), we estimate the following 2SLS  equations:
\begin{align}
M_{i} = \eta + \rho Democracy_i + X_i'\phi + \omega_i \\
\text{First Stage: }Democracy_i = \zeta + Z'_i\beta  + X_i'\delta + \upsilon_i.
\end{align}
This approach is similar to \citet{acemogluInstitutionalCauses}'s, which evaluates channels behind democracy' effects using similar 2SLS.

Table \ref{tab:channels} summarizes the results from this analysis.\footnote{We get similar results with alternative democracy indices, alternative weighting, and alternative sample definitions (excluding the US and China). Results are available upon request.} Panel A shows that democracy causes less severe responses at the 10th confirmed case of Covid-19. All estimates with controls show that a standard deviation increase in the democracy index causes the Containment Health Index to decrease by 0.4 to 0.5 standard deviations (see columns 2, 4, 6, 8, and 10). This effect of a standard deviation change in the democracy index is equivalent to around 25\% of the mean containment health index at the 10th confirmed case, suggesting that democracies cause significantly less severe containment policies at the beginning of the outbreak.

Democracy also narrows the coverage of containment policy, as documented in Panel B. For example, with log European settler mortality as an IV, we estimate that a standard deviation increase in the democracy index causes a decrease in the covered policy domains by 9.5 percentage points (s.e.= 2.7). All columns estimate that democracy causes narrower policy scopes in the pandemic's initial stage. 

On the other hand, democracy does not appear to cause slower responses. In fact, in Panel C, 8 out of 10 columns predict that democracy causes \textit{faster} responses. This leads to the bottom line that the severity and coverage of initial containment policies is a more important mechanism for the adverse effect of democracy than their speed.

In Appendix Table \ref{tab:causal-mediation}, we also use causal mediation analysis \citep{dippelCausalMediationInstrumentalVariables2020, imai2011unpacking} to measure how much the three potential channels explain the democracy impact. The results support our point that the severity and coverage of containment policies explain a large portion of democracy’s negative effect.