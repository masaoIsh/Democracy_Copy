Democracy dampens economic growth and causes more Covid-19-related deaths in 2020 through weaker and narrower containment policies. A likely reason for this result is that democracies tend to introduce weaker incentives and authority to enforce decisive, wide-ranging containment policies. Such containment policies are often unpopular, especially at the beginning of the pandemic when the scope of the crisis was still uncertain. This lack of popularity, or politicians' perception of it, may cause politicians in democracies to avoid such measures. Democracies also often lack the legal power to enforce lockdowns and other restrictive policies.

Our analysis leads to a variety of avenues for future work. First, it is important to update the analysis with better outcome data. 
%The data we currently use for GDP growth rates in 2020 are estimates by the IMF. We will update our analysis once official GDP growth rates are released. 
For example, we recognize reporting policy may affect the reported number of Covid-19-caused deaths. One potential solution is to look at data on excess mortality rates, such as the World Mortality Dataset \citep{karlinskyWorldMortalityDataset2021}. But the dataset is still limited in coverage and currently only covers excess mortality rates for about 80 countries. More conceptually, we plan to measure democracy's effects on other key aspects of policy performance, such as economic inequality and citizen's happiness. Finally, we need to see if the negative impact of democracy will result in geopolitical movements away from democracy. We leave these important directions to future work. 


The policy implication of our result is not straightforward. Needless to say, our analysis does not imply a general case against democracy, at least for two reasons. First, democracy per se has normative and procedural virtues, regardless of whether they result in good economic and health outcomes. Our analysis does not touch these normative and procedural values. More importantly, despite our finding on democracy's short-run, temporal impacts on outcomes during 2020, democracies may produce better outcomes in the long run. Our preferred interpretation of our findings is that there may be room for improvement in particular aspects of democracy in particular situations, so that governments can decisively and thoroughly take potentially unpopular, yet effective actions in the middle of emergencies like pandemics, infodemics, and natural disasters.



