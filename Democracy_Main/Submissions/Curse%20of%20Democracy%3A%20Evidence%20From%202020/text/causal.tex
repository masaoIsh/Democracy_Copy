
\subsection{IVs for Political Regimes} \label{subsec:instruments}

We cannot interpret the above relationship as causal, however. There are many omitted determinants of outcomes that also correlate with democracies. To identify the causal effect of democracy, we adopt five IV strategies. To be valid, the IVs must correlate with political regimes today (relevance) and correlate with Covid-19 outcomes solely through political regimes (exclusion restriction). Our choice of instruments considers several centuries of world history as follows. 

\textbf{European settler mortality IV.} European settler mortality is the mortality rate (annualized deaths per thousand mean strength) of soldiers, bishops, and sailors stationed in the colonies between the seventeenth and nineteenth centuries. \citet{acemogluColonialOriginsComparative2001} compile mortality data from earliest-available systematic records. Historians document that Europeans used mortality rates to decide where to settle \citep{curtinDeathMigrationEurope1989}. In colonies with inhospitable germs, Europeans did not want to settle and instead established extractive institutions. These extractive institutions' primary purpose was to transfer the colony's resources to the colonizer and did not provide checks and balances against government expropriation. In colonies with hospitable disease environments, Europeans settled and established inclusive institutions. Such institutions emphasized the protection of individual liberties and encouraged political participation. The effect of these colonial institutions persists to the present, as shown by their original study. 

\citet{acemogluColonialOriginsComparative2001} use this IV to show that inclusive institutions, which encompass the social, economic, legal, and political organizations of society, promote economic growth. 
Consistent with the above hypothesis by \citet{acemogluColonialOriginsComparative2001}, Figure \ref{fig:first-stage-european-settlers} and Appendix Table \ref{tab:first-stage} confirm that countries with higher European settler mortality have substantially lower levels of democracy today. This fact motivates us to use European settler mortality as an IV to estimate the causal effect of democracy. 

\textbf{Fraction speaking English or European, and the Frankel-Romer trade share as IVs.} The fraction speaking English or European is the fraction of a country's population speaking English or a major Western European language (French, German, Portuguese, and Spanish) as a mother tongue in 1992. The Frankel-Romer trade share is the predicted trade share of each country's economy, based on the gravity model of international trade.\footnote{\citet{frankelDoesTradeCause1999} first estimate a bilateral trade equation using the gravity model. Then, they aggregate the fitted values to estimate the trade share. The gravity model only considers population size and geographic characteristics such as country size and distance from other countries.} 
As \citet{hallWhyCountriesProduce1999} argue, a major feature of world history is the spread of Western European influence. This influence created an institutional and cultural background conducive to Western democracy. The language and trade variables are proxies for such Western influence. 
\citet{hallWhyCountriesProduce1999} use these IVs to show that social infrastructure positively affects productivity.\footnote{The original specification in their paper also uses absolute latitude as an IV. We do not use the latitude IV because it is likely to correlate with Covid-19 outcomes directly.} 

Indeed, the fraction of the population speaking a major European language positively correlates with Freedom House's democracy index, as reported in Figure \ref{fig:first-stage-fraction-european} and Appendix Table \ref{tab:first-stage}. Intuitively, the Frankel-Romer trade share measures how conducive the country's geography is to international trade. How open a country is to international trade correlates with social and political institutions. The first-stage effect of the Frankel-Romer trade share is less significant (Appendix Table \ref{tab:first-stage}), but we include it to avoid cherry-picking IVs and be as consistent with the original study as possible. 

\textbf{Legal origin IVs.} These IVs are dummy variables that are turned on if the country's legal origin is English, French, or German\footnote{\citet{portaLawFinance1998} also use a dummy variable for Scandinavian legal origin as an IV. We do not use it because it has little explanatory power (only applies to four countries in our sample), but adding it as an IV produces similar results. Results are available upon request.}. Many countries derived their legal systems from colonization by one of these European powers. Such legal origin determined the general legal infrastructure and influenced how the law protects civil liberties and political rights. By having separate dummies for legal origin, we allow them to have varying effects on political institutions today. With these IVs, \citet{portaLawFinance1998} show that British common-law brings about the strongest, and French civil-law the weakest, legal protections for investors, and stronger legal protections for investors promote financial development. The legal origin IVs turn out to be significant determinants of democracy in our setting, as shown in Appendix Table \ref{tab:first-stage}.

\textbf{The availability of crops and minerals as IVs.} Bananas, coffee, maize, millet, rice, rubber, sugarcane, and wheat are dummy variables that take the value 1 if a country produced the particular commodity in 1990. We code copper and silver as 1 if a country mined the mineral in 1990. According to \citet{sokoloffInstitutionsFactorEndowments2000}, certain commodities induced economies of scale and incentivized the use of slave labor, which led to extractive institutions. Meanwhile, other commodities encouraged production by middle-class family farmers, which induced inclusive institutions. Thus, the dummies for the ability to grow crops or mine minerals reflect historical agricultural endowments, which in turn reflect historical conditions for political regimes. Based on this IV, \citet{easterlyTropicsGermsCrops2003} show that geographic endowments affect development only through social and political institutions and that these institutions encourage economic growth.\footnote{Since Easterly and Levine's dataset only contains data for 71 countries, we extend their data as explained in Appendix \ref{subsubsec:easterly-data}.} 

\textbf{Past population density IV.} Population density in the 1500s is the number of inhabitants per square kilometer in the 16th century. The intuition behind this IV is that population density at the beginning of the colonial age determined colonial institutions' inclusiveness \citep{acemogluReversalFortuneGeography2002}. Sparse populations at the beginning of European expansion in the 16th century induced Europeans to settle and develop Western-style institutions, while larger populations made extractive institutions more profitable. The effect of these colonial institutions persists to the present. \citet{acemogluReversalFortuneGeography2002} use this IV to show that institutions have a positive effect on persistent economic growth.\footnote{They also use a measure for urbanization in the 1500s as an IV. We find that using this IV produces similar estimates to the estimates using population density in the 1500s as an IV. The results are available upon request.}

We are aware that none of these IVs is perfect. Each IV is likely to be threatened by its own mix of measurement errors, omitted variables, and exclusion violations. Our strategy is to use these five different IVs with the expectation that they work as robustness checks with each other.

\subsection{IV Estimation} \label{subsec:equation}
This section presents our main results. 
With the above IVs, we estimate democracy's impact by the following 2SLS regressions: 
\begin{align}
    Y_i &= \mu + \alpha Democracy_i + X^{'}_i \gamma + \epsilon_i \tag{1} \label{eqn:2sls-second}\\
    Democracy_i &= \zeta + Z^{'}_i\beta  + X^{'}_i \delta + \upsilon_i \label{eqn:2sls-first}
\end{align}
\noindent The second-stage equation (\ref{eqn:2sls-second}) is the same as Section \ref{sec:ols}'s OLS regression. The coefficient $\alpha$ represents the effect of the democracy measure $Democracy_i$ on $Y_i$, the outcome variable (GDP growth in 2020 or Covid-19-related deaths per million), conditional on a vector of country characteristics $X^{'}_i$ as controls. Given that $Democracy_i$ is far from randomly assigned, we instrument for $Democracy_i$ by each vector of IVs, $Z^{'}$, in the first-stage equation (\ref{eqn:2sls-first}).  

Does democracy cause worse economic and public health outcomes in 2020? Reduced-form figures using European settler mortality suggest so. Figures \ref{fig:reduced-gdp-logem} and \ref{fig:reduced-deaths-logem} show that higher European settler mortality causes lower levels of democracy, which cause higher GDP growth rates in 2020 and fewer deaths from Covid-19. 

Table \ref{tab:2sls} reports the 2SLS estimates of the effect of democracy, using each of the five IV strategies. They all indicate significant adverse effects of democracy. 
% Log European settler mortality
Columns 1 and 2 show our estimates using log European settler mortality as an IV. The first-stage regression in Appendix Table \ref{tab:first-stage} column 1 shows that higher log European settler mortality results in lower levels of democracy today, with a coefficient of -0.6 (s.e. = 0.2) and an F-statistic of 10.1. The corresponding 2SLS regression estimates in Panel A's column 1 show that a standard deviation increase in the democracy measure causes a 3.1 (s.e. = 0.7) percentage-point decrease in GDP in 2020 and 440.5 (s.e. = 87.6) more Covid-19-related deaths per million. Once we account for this effect, countries in Europe, North America, or South America do not have significantly worse Covid-19 outcomes (Table \ref{tab:remove-democracy-effect}). In column 2, we control for climate, population density, population aging, and diabetes prevalence. The magnitudes of the coefficients change little. The estimates are -2.6 (s.e. = 0.7) percentage points for GDP growth rates and 494.0 (s.e. = 120.0) for Covid-19-related deaths per million. 

%To see whether these 2SLS estimates make quantitative sense, we again compare Egypt and Spain.\footnote{Egypt has the 15th percentile of the normalized democracy measure in this sample (normalized score = 0.7). Spain has the 85th percentile score (normalized score = 3.1).} Our 2SLS estimates in column (1) imply the 2.4 difference in the normalized democracy index translates into a -10.6 difference in GDP growth rates in 2020 and a 1027 difference in Covid-19-related deaths per million. Given that the average GDP growth rate in 2020 is -5.7\% and the average of Covid-19-related deaths per million is 285, the estimates imply a large effect.

To check whether the above results are sensitive to the choice of IVs, columns 3 and 4 use as IVs the fraction speaking English or European and the log Frankel-Romer trade share. We continue to find a negative effect of democracy. %Panel B's column 3 shows that, although the coefficient on the log Frankel-Romer trade share IV is statistically insignificant, the three IVs jointly explain 60\% of the variation in the democracy index. 
The corresponding 2SLS estimates in column 3 are -2.7 (s.e. = 0.7) percentage points for GDP in 2020 and 416.9 (s.e. = 127.8) for Covid-19-related deaths per million. Even with controls, the estimates stay almost the same. 

% Legal origin
% The overall pattern remains the same for the legal origin IVs in columns 5 and 6. Panel B's column 5 shows that British legal origin positively correlate with the democracy index, while the effect of French and German legal origin is less clear. The legal origins explain 20\% of the variation in democratic levels. Panel A's corresponding 2SLS estimates are -5.5 (s.e. = 1.3) for GDP growth rates, and 109.6 (s.e. = 116.5) for Covid-19-related deaths per million. The regression with controls in column 6 produce similar results.

The overall pattern remains the same for the legal origin IVs in columns 5 and 6. Appendix Table \ref{tab:first-stage}'s column 5 shows that British legal origin positively correlates with the democracy index, with an F-statistic of 6.8. The corresponding 2SLS estimates are -3.5 (s.e. = 1.4) for GDP growth rates, and 550.4 (s.e. = 335.6) for Covid-19-related deaths per million. The regression with controls in column 6 produces similar results.

% The ability to grow crops as an IV
% Columns 7 and 8 use dummies for the ability to grow certain crops and mine minerals as IVs. The results are similar to those in previous columns. The first-stage regression results in Panel B's column 7 suggest that these variables alone are unreliable IVs. Yet, once we control for baseline covariates in column 8, the $R^2$ increases to 0.6, with an F-statistic of 7.6. The coefficients in column 8 are -4.8 (s.e. = 1.3) for GDP growth rates in 2020, and 320.3 (s.e. = 82.7) for Covid-19-related deaths per million.

Columns 7 and 8 use dummies for the ability to grow certain crops and mine minerals as IVs. The results are similar to those in the previous columns. The first-stage regression has an F-statistic of 8.8. The corresponding coefficients in column 7 are -2.5 (s.e. = 0.7) for GDP growth rates in 2020, and 297.4 (s.e. = 90.0) for Covid-19-related deaths per million. Controlling for baseline covariates in column 8 results in similar estimates with smaller standard errors. 

%in Appendix Table \ref{tab:first-stage}'s column 7 shows that the IVs explain 60\% of the variation in democracy levels

% Log population density in the 1500s
Finally, we use population density in the 1500s as an IV in columns 9 and 10. The estimates are consistent with our baseline results. The first-stage relationship's F-statistic in column 9 shows that log population density in the 1500s alone is a weaker IV for democracy. Yet, the F-statistic increases to 5.3 after we control for other control covariates. The corresponding 2SLS estimates are -2.1 (s.e. = 0.7) for GDP growth rates in 2020 and 486.4 (s.e. = 137.9) for Covid-19-related deaths per million. 

In general, the 2SLS estimates in Panel A are larger in magnitude than the OLS estimates in Panel B. This suggests that there is omitted variable bias in our OLS estimates. One potential omitted variable is the quality of public health systems. It is likely to be positively correlated with democracy, negatively correlated with Covid-19 deaths, and positively correlated with GDP growth rates in 2020. Another potential explanation is measurement error. In reality, the democratic institutions that matter for performance in the pandemic are complex, and no single measure can capture democracy levels precisely enough. Such measurement error may create attenuation bias in our OLS estimates. 