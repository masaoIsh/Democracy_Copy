
\subsection{2SLS estimation}

    Here, we treat the democracy variable $Democracy_i$ as endogenous, and model it as:
    \begin{equation}
        \label{eqn:2sls}
        Democracy_i = \zeta + \beta log M_i + X_i \delta + \upsilon_i
    \end{equation}
    where $M_i$ is the settler mortality rate in 1,000 mean strength. The exclusion restriction is that this variable does not appear in Equation \ref{eqn:ols}. 
    
Figure \ref{fig:reduced-form} shows the reduced form figures. What we observe is consistent with the relationship suggested by OLS. It appears that higher European settler mortality caused lower levels of democracy and this on average led to better performance in the Covid-19 pandemic. Higher mortality rates are ultimately associated with lower Covid-19 deaths per million and positive economic growth between 2019 and 2020. 
    
Panel B in Table \ref{tab:2sls} shows the negative first-stage relationship between (log) settler mortality and current democracy levels for the sample in which we observe European settler mortality rates. This negative relationship is statistically significant (mostly at the 1\% level, using heteroscedasticity-robust standard errors). The F-statistics in columns (1), (3), (5) and (7)\footnote{The regressions in (1) and (5), as well as (3) and (7) are identical to each other} are 26.2 and 29.2 - which is well above the critical value of 10 suggested by \textcite{staiger-stock} to ensure that the maximum bias in the IV estimators are less than 10\%. 

Moreover, reassuringly, these first-stage estimates align with the hypothesis that higher levels of settler mortality caused European colonizers to set up more ``extractive" states without European-style political systems, which has persisted over time and led to lower current levels of democracy. This negative relationship between settler mortality and the democracy index is exactly what we observe in Figure \ref{fig:first-stage}, where higher levels of European settler mortality rates are generally associated with lower levels of democracy. 

Our main 2SLS estimate (in Panel A, columns (1) and (5)) of the impact of democracy on Covid-19-related outcomes is a 370.6 increase in Covid-19-related deaths per million and a 3.8 percentage decrease in annual GDP between 2019 and 2020 per a standard deviation increase in the democracy measure. 
Do these 2SLS estimates make quantitative sense? Does it imply that differences in democratic levels can explain a significant fraction of differences in Covid-19-related outcomes across countries? Let us again compare two ``typical" countries with high and low democracy indices, Costa Rica and Venezuela. Our 2SLS estimate of 370.6 deaths per million and -3.8 in annual GDP change per standard deviation implies that the approximate 2.4 standard deviation difference in the democracy index should translate to an increase of around 890 Covid-19 deaths per million and a decrease in total GDP between 2019 and 2020 of around 9\% in 2020. In practice, the presence of measurement error complicates this interpretation, because some of the difference between Costa Rica and Venezuela's democracy measure may reflect measurement error. Therefore, the nearly 900 increase in deaths per million and 9\% decrease in GDP over 2019-20 per standard deviation is an upper bound. However, considering again that average deaths per million is 237 and the average GDP loss is 3.1\%, the estimates (increase of 370.6 deaths per million and GDP loss of 3.8\% per standard deviation) imply a substantial, but not implausibly large, effect of democratic levels on Covid-19 performance - both in terms of Covid-19 mortality and the economy. 
    
To deal with the concern that China and US may be driving these results, in columns (2) and (4), we have excluded those two countries from the sample. Reassuringly, the exclusion of these two countries does not change our main interpretation  - the estimates remain statistically significant and large in absolute terms. This documents that our results are not driven by US and China.
    
Columns (3) and (7) show that adding controls for climate, wealth and density, does not change the basic relationship; the democracy measure's coefficient is now 296.8 per million in terms of mortality with a 2SLS robust standard error of 129.5 and -5.7\% in terms of economic growth with a 2SLS robust standard error of 2.1. Interestingly, none of the control variables concerning climate (absolute latitude, mean temperature and precipitation) are statistically significant, implying that, once democracy has been controlled for, climate factors are not significant determinants of performance in the pandemic. Finally, in columns (4) and (8) we repeat the regression with controls on the sample excluding the US and China. Although the robust 2SLS standard errors for the estimate of the effect of democracy on Covid-19 mortality is larger, the estimates remain in line with our baseline results.  

That most of our 2SLS estimates in Panel A are larger (in absolute terms) than the OLS estimates reported in Panel C points to two main potential reasons: the existence of measurement error of ``true" democracy and the existence of omitted variables in the OLS regression that are negatively related to Covid-19 deaths per million and positively related to GDP change between 2019 and 2020. In terms of measurement error, we recognize that in reality the set of democratic institutions that matter for performance in the pandemic is very complex, and what matters for current performance is presumably not only democratic levels today, but also the state of democracy in the past. In any case, this is a classic case of measurement error and hard to resolve completely. On the other hand, the existence of omitted variables that are negatively related to Covid-19 mortality and positively related to economic growth is interesting and calls for future analysis. 

\subsection{Alternative IVs}

As introduced in Section \ref{sec:intro}, we also consider the use of IVS introduced by Hall and Jones, namely the fraction of population speaking English as a mother tongue, the fraction of population speaking one of the five primary Western European languages (English, French, German, Portugese and Spanish) as a mother tongue and the Frankel-Romer predicted trade share. \footnote{In their paper, they also use absolute latitude as one of their IVs, but we do not use include it into our 2SLS regression because it is unclear whether it satisfies the exclusion restriction.} Hall and Jones hypothesize that the two variables relating to language are correlates of the extent of Western European influence. Since the extent of influence varied across countries, it acts as identifying variation which we can plausibly take to be exogenous. Furthermore, they also use as an IV the variable constructed by \textcite{frankelromer}, the log predicted trade share of an economy. It is based on the gravity model of international trade and considers only a country's population and geographical features. Hall and Jones use these variables as IVs for social institutions, and given that social institutions and democracy levels are highly correlated, we think it is reasonable to adopt their IVs in our analysis. 
Table \ref{tab:2sls-instruments} compares the results for the estimation of the 2SLS regression on the total number of deaths attributed to Covid-19 in 2020 when using the different the IV strategies of Acemoglu et al and Hall and Jones. 

Panel B's columns (3), (4), (7) and (8) show the results of first stage regressions on the democracy index using the IVS proposed by Hall and Jones (regressions in columns (3) and (7), (4) and (8) are identical to each other). Compared to the F-statistics of log European settler mortality (26.2 without controls and 29.2 with controls), the F-statistics 14.7 (without controls) and 26.1 (with controls) are smaller, implying that the IVs are relatively weak. However, the F-statistics remain larger than the threshold  of 10 required to ensure that the maximum bias in IV estimators to be less than 10\% \parencite{staiger-stock}. Thus, we believe that, although the point estimates may be less reliable, they still aid in confirming the validity of our baseline results. 

The 2SLS estimates using Hall and Jones' IVs (in Panel A's columns (3), (4), (7) and (8)) are consistent with the results obtained using European settler mortality as an IV. The regressions without controls (in Panel A, columns (3) and (5)) gives estimates of 554.3 for Covid-19-related deaths and -2.7 for GDP change between 2019 and 2020 per standard deviation. The addition of control variables in columns (4) and (6) gives consistent results, with estimates of 565.4 for deaths and -4.2 for GDP change. With the exception of column (7), all of these estimates are statistically significant at the 10\% level. In summary, the use of alternative IVs does not change the fact that we continue to estimate a statistically significant causal effect of democracy on Covid-19 performance. 

