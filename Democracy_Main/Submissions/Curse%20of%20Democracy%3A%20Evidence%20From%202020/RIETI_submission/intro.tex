
Does democracy contribute to economic growth and safety of life? %Although a consensus is still to be reached, democracy and its relationship to the successfulness of states has long been of considerable debate. 
The long-standing debate has garnered yet even more attention with the outbreak of the Covid-19 pandemic. Many cannot help but observe the large number of deaths and economic devastation in democratic countries such as the US and France, which make a stark contrast with non-democratic states such as China. %, which have experienced relative success in mitigating the pandemic's impact. 
%However, the question of whether or not democracy has a causal effect on worse performance in the Covid-19 pandemic has not yet been fully inspected - That is the endeavour we wish to take on in this paper. 
%By carefully inspecting democracy's role in the Covid-19 pandemic, we contribute to the wider literature on the relationship between democracies, economic growth and safety of life, and prompt timely discussion on whether, when, and why democracies fail. 
Indeed, Figure \ref{fig:ols} shows a strong global correlation between the Economist Intelligence Unit’s Democracy Index 2019 and Covid-19-related outcomes. We specifically look at the number of Covid-19-caused deaths per million during 2020 and the GDP growth from 2019 to 2020. 
Worse performance in both outcomes is associated with democracy. 

%differences in the level of democracy create performance differences in the handling of the Covid-19 pandemic. 

%By democratic levels we mean the extent to which the country's citizens have control over the functions of their countries' institutions and government policies. This definition not only considers the functionality of electoral systems and the accountability of government officials through institutional checks and balances but also ``softer" aspects such as active political participation, pluralist political cultures, and the protection of civil liberties. In measuring the performance of countries in the pandemic, we use mortality rates and economic growth levels. More specifically, we look at the total number of deaths attributed to Covid-19 per million between January 1st 2020 to December 31st 2020 as well as the total percentage change in GDP between 2019 and 2020. %Given these definitions, we hypothesize that democracies that incorporate the views of citizens (not necessarily the majority view) are ineffective at imposing the politically difficult but critically important policies needed to handle crises such as the current Covid-19 pandemic and thus face higher mortality rates as well as larger economic damage. 
    
This fact motivates us to study whether this association has any causal meaning. To identify the causal effect of democracy, we adopt several exogenous sources of variation in political systems. 
We first utilize the instrumental variable (IV) used by \textcite{ajr}, namely, European settler mortality. They justify  this IV on the following grounds: (1) different colonization policies adopted by European settlers fell somewhere in the spectrum between ``extractive" states, where institutions did not implement ``European" ideas such as the the protection of private property or checks and balances against government powers, and ``neo-European" states, where institutions replicated European institutions with strong emphasis on private property and checks against government power. (2) Such colonization strategies were influenced by the mortality of European settlers, which was a proxy for the favorability of the disease environment for Europeans. Where the conditions were less favorable for Europeans, they set up more ``extractive" states. Where the conditions were more favorable, they set up ``neo-European" states. (3) The effect of colonial states and institutions persisted to the present day. %Building on these premises, in this paper, we assume that European settler mortality functions as a reliable source of exogenous variation that will help us identify whether democracy has a significant causal effect on Covid-19-related outcomes. 

% explanation of results

Our 2SLS estimates show a significant impact of democracy on worse Covid-19-related outcomes. 
A standard deviation increase in the democracy index causes 370.6 more Covid-19-related deaths per million and a 3.8 percentage point decrease in annual GDP between 2019 and 2020. To provide a benchmark, a standard deviation change in the democracy index corresponds to the difference between Slovakia and Sweden. Our estimates are statistically significant at the 1\% level. We also confirm that our result holds even if excluding China and the US from the sample. Neither introducing controls nor weighting countries differently changes the baseline results. 

We also adopt \textcite{hj}'s IVs based on the geographical and linguistic characteristics of a country. Hall and Jones hypothesize that these characteristics affect the extent to which a country has been influenced by Western Europe's democratic political systems. Adopting their alternative IV produces similar results to the baseline results. \\
% explanation of results 

Our work is at the intersection of two strands of the literature: a longstanding literature on the relationship among democracy, economic growth and safety of life, and an emerging literature on the economics of pandemics. We integrate them to find that democracy causes worse outcomes in Covid-19 mortality and economic growth. 

The effect of democracy is difficult to analyze, due to omitted variable biases, measurement errors, as well as limited data size. A consensus is hard to reach \parencite{meta}. 
Some scholars claim that the effect of democracy on economic growth and safety of life is negligible. \textcite{barro}, one of the first to estimate the effect of democracy on economic outcomes, finds that democracy has a small negative effect on economic growth. \textcite{tavares}, like Barro, use cross-country regressions to find a weak negative effect. Studies using panel data also find no significant effects of democracy on growth and general prosperity \parencite{burkhart, giavazzi}. 

At the same time, other scholars argue the opposite. Studies point out that democracies are more responsive to the public's demands in areas such as education, justice, health and public goods, all of which contribute to long-term growth \parencite{edu-dem-growth, benabou, baum-lake2001, baum-lake2003, rodrik1998}. 
Other studies stress the ability of democracies to provide more stability \parencite{sah}, which is closely related to economic growth \parencite{quinn-woolley}. More recent works rely on panel data to claim that although democratization is initially associated with low growth rates, democratic nations tend to experience long-term economic growth \parencite{dem-growth, ajr2014}.

% Moreover, studies show that democracies induce a business-friendly climate of liberty and free-flowing information, making citizens more motivated to work and invest \parencite{north}. 

The second literature we build on is the literature on the economics of pandemics. %As the global death toll of the Covid-19 pandemic exceeds 2 million, the complex issue of how governments, civilians and the economy interrelate in the Covid-19 pandemic and what factors determines the different levels of success is an issue of grave importance. Naturally, numerous studies attempt to uncover the numerous factors and dynamics at play. 
Some studies focus on trade-offs at the individual level. For example, \textcite{civil-liberty-survey} study the trade-off between civil liberty and public health using large-scale surveys. They find great heterogeneity in attitudes. While only 5\% of respondents in China are unwilling to sacrifice any rights even during times of major crisis, nearly four times as many respondents in the United States are willing to do so. In terms of the trade-off between health and the economy, \textcite{health-wealth} use surveys in the US and UK to find that people prioritize saving lives but this valuation changes as economic losses increase. 

Others focus on the interaction between governments and civilians. For example, \textcite{politics-survey} focus on the political and electoral consequences of the Covid-19 pandemic and find that governments are punished in terms of political approval when Covid-19 infections accelerate, particularly in the absence of effective lockdown measures. Some also studied how obedience to travel restrictions or compliance with social distancing during the Covid-19 crisis differs according to factors such as collectivist culture, social capital, trust in governments and political beliefs \parencite{frey, social-distancing, social-capital, covid-communication}. 

We organize this paper as follows. Section \ref{sec:data} describes our data and provides descriptive statistics.  Section \ref{sec:ols} analyzes the correlation between Covid-19-related outcomes and democracy. Section \ref{sec:causal} presents our main 2SLS results. Section \ref{sec:robust} shows the results under alternative specifications, and Section \ref{sec:conclusion} concludes. 